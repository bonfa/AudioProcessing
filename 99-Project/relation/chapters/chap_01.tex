\chapter*{Introduzione}

Per suonare musica di alta qualità, è necessario avere strumenti perfettamente accordati.
Come tutti i musicisti sanno, accordare uno strumento può essere difficoltoso, soprattutto per un orecchio non perfettamente allenato.
In questa relazione, si discute la realizzazione in Matlab di un software che rappresenta un accordatore digitale, cioè un programma che registra il suono prodotto dalla corda di una chitarra, lo analizza calcolando la frequenza fondamentale e rappresenta la differenza rispetto a una frequenza di riferimento, che è scelta tra le frequenze fondamentali delle corde dello strumento.
Questo lavoro rappresenta il progetto del corso di \emph{Digital Audio Processing} della Facoltà di Ingegneria dell'Università degli Studi di Brescia.

La nota prodotta da una corda della chitarra non è composta da una singola frequenza ma da un certo numero di armoniche. 
Il \emph{timbro}, cioè l'insieme delle armoniche prodotte da un determinato strumento, è ciò che distingue i diversi strumenti musicali tra loro quando suonano la stessa nota.
Nel momento in cui una nota viene suonata, l'accordatore deve stabilire la frequenza fondamentale corrispondente.
Una delle parti principali di questo lavoro consiste proprio nel calcolo della frequenza fondamentale a partire dallo spettro del segnale della nota.

Un'altro argomento fondamentale trattato è il dimensionamento dei due parametri frequenza di campionamento e ampiezza del frame temporale registrato al fine di garantire il funzionamento in tempo reale del software mantenendo una precisione accettabile in frequenza.
Questi due requisiti richiedono valori di parametri in contrasto l'uno con l'altro.
Di seguito si discute ampiamente questo fatto e si propone una soluzione per aumentare la risoluzione ottenuta senza modificare i due valori scelti.

L'analisi del rumore in ingresso, prima operazione realizzata, e la realizzazione dell'interfaccia grafica del software, completano il lavoro svolto.
