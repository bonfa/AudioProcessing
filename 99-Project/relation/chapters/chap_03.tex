\chapter*{I Parametri Fondamentali}\label{cap:parametri}

La seconda operazione da effettuare per la creazione dell'accordatore è la definizione dei parametri di base, cioè la frequenza di campionamento e la durata dell'intervallo temporale.  

Uno dei requisiti fondamentali di un accordatore è la capacità di lavorare in tempo reale. 
In questo contesto, il termine indica che il tempo trascorso tra la ricezione dell'input e la comunicazione dell'output deve essere percepito come istantaneo dall'utente. 
Le operazioni di analisi della frequenza fondamentale della corda, calcolo della distanza tra la frequenza ottenuta e la frequenza di riferimento e comunicazione della risposta, quindi, devono essere effettuate in un intervallo temporale ridotto, il cui limite massimo è stato impostato a 1 secondo dal progettista.
La conseguenza di questo requisito è che la frequenza di campionamento deve essere la più piccola possibile. 
Tenendo conto del teorema del campionamento e supponendo che l'intervallo di frequenze compreso tra 0 e \mbox{4 kHz} contenga informazione sufficiente per rilevare la frequenza fondamentale di ciascuna corda della chitarra, la minima frequenza di campionamento presa in considerazione è di 8 kHz.

Il secondo requisito fondamentale di cui tenere conto nella definizione dei parametri è la precisione dell'analisi della frequenza.
È desiderabile infatti avere un \mbox{JND} almeno pari a quello dell'orecchio umano. 
Questa quantità rappresenta la minima variazione che la frequenza di un suono deve subire affinché il suono risultante venga percepito come distinto dall'originale. 
Il \mbox{JND} è una misura della precisione dell'apparato uditivo umano e dipende dalla frequenza dei suoni.
Per toni complessi che hanno frequenza inferiore ai 500 hz il è di circa 1 hz \todo{riferimento}. 
Dal momento che tutte le frequenze fondamentali delle stringhe della chitarra sono inferiori ai 500 hz, la precisione di 1 hz può essere considerata accettabile.
Fissate la frequenza di campionamento e la risoluzione desiderata, il numero di campioni da analizzare è calcolato dalla seguente formula.

\vspace{0.2cm}
\centerline{\textit{Numero di Campioni = Frequenza di Campionamento / Risoluzione}}
\vspace{0.2cm}

Sostituendo i valori di 1 hz per la risoluzione e di 8 kHz per la frequenza di campionamento, il numero di campioni al secondo risulta essere 8000. 
Questo determina che l'intervallo temporale in cui analizzare nuovi campioni deve essere di un secondo.

I parametri utilizzati per l'accordatore sono i seguenti: frequenza di campionamento di 8 kHz e finestra temporale totale di 1 secondo. 
Nell'intervallo temporale definito, che, come si vedrà nel capitolo dedicato all'interfaccia grafica, è scandito da un timer, non deve essere realizzata solo l'operazione di estrazione dei campioni ma devono essere svolte anche le operazioni di analisi della frequenza del suono, di calcolo della distanza dalla frequenza fondamentale e aggiornamento dell'interfaccia grafica. 
Queste operazioni restringono l'intervallo temporale in cui si estraggono campioni a un intervallo variabile tra 0,65 e 0,89 secondi, con la conseguente riduzione della risoluzione tra i valori 1.56 e 1,12 hz.

Ottenere risoluzioni maggiori a quelle determinate variando solamente frequenza di campionamento e intervallo temporale e rispettando il vincolo di tempo reale definito risulta essere molto difficoltoso. 
A causa del limite di 1 secondo definito dallo sviluppatore, non è possibile estendere la finestra temporale ulteriormente.
Aumentare la frequenza di campionamento, invece, aumenterebbe il numero di campioni da elaborare con il conseguente aumento del tempo di elaborazione e la riduzione dell'intervallo temporale su cui analizzare campioni finendo con il causare una diminuzione di risoluzione.

La soluzione adottata per aumentare la risoluzione è quella dell'interpolazione.
Questa operazione è descritta nella sezione \todo{riferimento}.

