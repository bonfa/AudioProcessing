\section{Interpolazione}\label{cap:interpolazione}

Come accennato nel capitolo \ref{cap:parametri}, i valori della frequenza di campionamento e del numero di campioni estratti per frame non garantiscono la precisione desiderata.
La soluzione introdotta per aumentare tale precisione è interpolare tramite un metodo di tipo \emph{spline} i valori attorno al massimo trovato.
La funzione opera su un intorno di ventuno punti, centrato attorno al massimo della funzione.

	\begin{figure}[h]
	  \begin{center} 
	    \includegraphics[width=\textwidth*\real{0.9}]{images/ch_05/interpolazione.jpg}
	  \end{center} 
	  \caption{\textit{Interpolazione per migliorare la ricerca del massimo}}  
	  \label{fig:interpolazione}
	\end{figure}

La figura \ref{fig:interpolazione} mostra un esempio dell'interpolazione realizzata. 
I punti evidenziati in rosso, rappresentano il prodotto delle versioni sotto-campionate dello spettro del segnale audio.
La linea verde che li congiunge, rappresenta un'interpolazione lineare di tali punti mentre la linea blu l'interpolazione spline, realizzata al fine di trovare la frequenza fondamentale con una maggiore precisione.

L'operazione di ricerca del massimo viene ripetuta sulla nuova funzione ottenuta e l'ascissa del punto trovato viene utilizzata come frequenza della nota prodotta dalla corda della chitarra.

Al fine di eliminare falsi positivi, cioè massimi che possono essere rilevati in momenti in cui non si sta suonando alcuna corda tra quelle della chitarra, è fissata una soglia in ampiezza al di sotto della quale la frequenza del massimo non è ritenuta una frequenza di pitch.
Tramite l'algoritmo \mbox{HPS} è facile distinguere quest'ultimo caso.

In presenza di una nota a una determinata frequenza, la frequenza fondamentale risulta essere molto amplificata.
Ciò deriva dal prodotto di 5 massimi locali nelle diverse funzioni scalate del segnale originale.
Nel caso di rumore, invece, le diverse frequenze pur moltiplicandosi tra loro, non subiscono l'effetto di amplificazione che subisce la frequenza fondamentale nel primo caso.

La soglia minima di ampiezza al di sopra della quale la frequenza viene considerata fondamentale è fissata a 84 dB.
Si ricorda che il segnale su cui è applicata tale soglia è il risultato della moltiplicazione di cinque segnali, di conseguenza, in presenza di un pitch il massimo della funzione è sempre al di sopra di tale soglia.


