\chapter{Conclusioni}\label{cap:conclusioni}

In questa relazione viene discussa la creazione di un accordatore digitale per chitarre.
L'operazione più importante da realizzare per raggiungere tale obiettivo è riconoscere la frequenza fondamentale delle note prodotte da una chitarra.
Il risultato di questa operazione è influenzato dal rumore in ingresso e da due parametri fondamentali, frequenza di campionamento e dimensione del frammento temporale.
Questi ultimi due valori dipendono a loro volta dai requisiti temporali e di precisione richiesti.

L'algoritmo utilizzato per riconoscere la frequenza fondamentale delle note musicali si basa sull'algoritmo Harmonic Product Spectrum, che moltiplica diverse versioni scalate dello spettro del segnale e calcola una prima stima della frequenza fondamentale come massimo della funzione ottenuta.
Per aumentare la precisione della risultato, viene effettuata un'interpolazione in un'intorno del massimo trovato.
Una seconda operazione di massimo viene calcolata sulla funzione interpolata e il risultato risulta essere la frequenza fondamentale.
Tale valore viene utilizzato per calcolare la distanza dalla frequenza di riferimento, che è impostata dall'utente tramite l'interfaccia grafica. 

Le prestazioni del software sviluppato soddisfano le attese.
La sua precisione è influenzata principalmente dall'utilizzo della trasformata di Fourier discreta, che presenta un passo di discretizzazione costante nel dominio delle frequenze.
Se si valuta la precisione dal punto di vista dell'orecchio umano, essendo essa una funzione logaritmica della frequenza, metodi con un passo di discretizzazione con questo andamento, a parità di parametri, garantiscono una precisione maggiore rispetto all'algoritmo sviluppato.



