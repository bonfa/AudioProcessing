\chapter{Conclusioni}\label{cap:conclusioni}

In questa relazione viene discussa la creazione di un accordatore digitale per chitarre.
L'operazione più importante da realizzare per raggiungere tale obiettivo è riconoscere la frequenza fondamentale delle note prodotte da una chitarra.
Il risultato di questa operazione è influenzato dal rumore in ingresso e da due parametri fondamentali, frequenza di campionamento e dimensione del frammento temporale.
Questi ultimi due valori dipendono a loro volta dai requisiti temporali e di precisione richiesti.

L'algoritmo utilizzato per riconoscere la frequenza fondamentale delle note musicali si basa sull'algoritmo Harmonic Product Spectrum, che moltiplica diverse versioni scalate dello spettro del segnale e 

Vengono discussi, inoltre, i principali fattori che influenzano l'analisi di tale parametro, cioè il rumore in ingresso, la frequenza di campionamento, la lunghezza del frammento temporale e l'andamento logaritmico della percezione della frequenza. 
Infine viene presentata l'interfaccia grafica, che fornisce una rappresentazione intuitiva per l'utilizzo del programma sviluppato.


