\chapter{Differenza della distanza dalle frequenze di riferimento}\label{cap:distanza}

La frequenza della corda viene confrontata con la frequenza di riferimento desiderata per valutare la differenza dalla frequenza corretta.
Come scegliere la frequenza di riferimento è descritto nel capitolo che descrive l'interfaccia grafica del software.
La tabella \ref{tab:frequenze_riferimento} contiene per ciascuna corda, il numero, la nota prodotta nel caso in cui la corda venga pizzicata e la relativa frequenza di riferimento.

	\begin{table}[h]
	\center
	\begin{tabular}{|l|c|r|}
		\hline
		Corda	& 	Nota    & Frequenza (hz) \\
		\hline
		6	&	E	&	82.4     \\
		5	&	A	&	110      \\
		4	&	D	&	146.8    \\
		3	&	G	&	196      \\
		2	&	B	&	246.9    \\
		1	&	E	&	329.6    \\
		\hline
	\end{tabular}
	\caption{\textit{Tabella contenente la frequenza della nota di ciascuna corda della chitarra}}
	\label{tab:frequenze_riferimento}
	\end{table}

La distanza tra la frequenza di riferimento e la frequenza del segnale è calcolata con una semplice sottrazione.
In questo caso, l'andamento logaritmico della percezione delle frequenze può essere trascurato in quanto per il range di frequenze preso in considerazione l'ampiezza della banda critica rimane costante a 100 hz. 
