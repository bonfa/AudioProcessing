\section{Differenza della distanza dalle frequenze di riferimento}\label{cap:distanza}

La frequenza della corda viene confrontata con la frequenza di riferimento desiderata per valutarne la distanza.
La scelta della frequenza di riferimento è descritta nel capitolo \ref{cap:interfaccia}.
La tabella \ref{tab:frequenze_riferimento} contiene per ciascuna corda, il numero di riferimento, la nota prodotta nel caso in cui la corda venga pizzicata e la relativa frequenza di riferimento.

	\begin{table}[h]
	\center
	\begin{tabular}{|l|c|r|}
		\hline
		Corda	& 	Nota    & Frequenza (Hz) \\
		\hline
		1	&	E	&	329.6    \\
		2	&	B	&	246.9    \\
		3	&	G	&	196      \\
		4	&	D	&	146.8    \\
		5	&	A	&	110      \\
		6	&	E	&	82.4     \\		
		\hline
	\end{tabular}
	\caption{\textit{Tabella contenente la frequenza della nota di ciascuna corda della chitarra \cite{giordano2009reasoning}}}
	\label{tab:frequenze_riferimento}
	\end{table}

La distanza tra la frequenza di riferimento e la frequenza del segnale è calcolata con una semplice sottrazione.
In questo caso, l'andamento logaritmico della percezione delle frequenze non viene considerato in quanto la precisione dell'accordatore è stata definita in hertz.
